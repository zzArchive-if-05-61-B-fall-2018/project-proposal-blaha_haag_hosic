\documentclass[12pt]{scrartcl}
\usepackage{ucs}
\usepackage[utf8x]{inputenc}
\usepackage[T1]{fontenc}
\usepackage[english]{babel}
\usepackage{setspace}
\usepackage{floatrow}
\usepackage[table]{xcolor}
\usepackage{graphicx}
\usepackage{lmodern}
\usepackage[automark]{scrpage2}
\usepackage{geometry}  
\usepackage{amssymb}
\usepackage{amsthm}
\usepackage{epstopdf}
\usepackage{caption}
\usepackage{floatrow}
\usepackage[table]{xcolor}
\renewcommand{\baselinestretch}{1.15} 
\newcolumntype{L}[1]{>{\raggedright\let\newline\\\arraybackslash\hspace{0pt}}m{#1}}
\pagestyle{scrheadings}
\clearscrheadfoot
\ihead[]{\headmark}
\ifoot[]{\author}
\ofoot[]{\pagemark}
\setheadsepline[\textwidth]{0.1pt}
\setkomafont{pageheadfoot}{\sffamily}
\setkomafont{pagenumber}{\bfseries}

\DeclareGraphicsRule{.tif}{png}{.png}{`convert #1 `dirname #1`/`basename #1 .tif`.png}

\title{Manual for Rest-Server}
\author{Johann Haag, Tarik Hosic und Simon Blaha}
\date{20.12.2018, Leonding}


\begin{document}
    \maketitle
    \begin{flushleft}
    \begin{tabular}{|l|l|}
    \hline
    Project Name & Smart Organizer \\ \hline
    Project Leader & Simon Blaha \\ \hline
    Version & 1.0\\ \hline
    Document state & In process \\ \hline
    \end{tabular}
    \end{flushleft}

    \pagebreak
    \tableofcontents
    \pagebreak

    \section{Installation}
    \subsection{Aufsetzen der MongoDB}
        Eine MongoDB kann standardmäßig auf den Computer installiert werden. Es bietet sich auch die Möglichkeit an, die MongoDB per Docker zu betreiben.
        Wenn sie die MongoDB auf den Rechner installieren möchten, sehen sie folgenden Link .....
        Falls sie Docker verwenden möchten, dann sehen sie auf folgender Seite nach ....

    \section{Inbetriebnahme des Rest-Servers}
        Um den Server zu starten gibt es mehrere Wege. 
        Zum einen können sie per CMD in das Root-Verzeichnis des Servers wechseln und node index.js ausführen oder 
        sie führen das Script run_server.sh aus. Hierbei wird jedoch die Skriptsprache Bash benötigt.
        Bei der Inbetriebnahme werden automatisch die benötigten Collections auf der MongoDB erstellt. 
        Es wird im Log ausgegeben, ob ein Fehler aufgetreten ist.
        Der Server kann einfach mit STRG+C im Terminal beendet werden.

    \section{Konfiguration}
        In der Conifg.json Datei des Servers können Einstellungen des Servers geändert werden.
        So kann hier die Portnummer sowie der Pfad zur MongoDB geändert werden.
        Bei der Standardinstallation ist immer Localhost und der standard Port für MongoDB eingetragen.

    \section{Testen der Installation}
        Im Ordner des Rest-Servers liegt ein Script namens test_rest_server.sh vor.
        Dieser Script kann nur auf Betriebssystemen verwendet werden, die das Programm ... installiert haben sowie die Skriptsprache Bash unterstützen.
        Wenn sie diesen Script verwenden, werden Testdaten an den Rest-Server geschickt und die abgefragt.
        Deswegen ist es gut, wenn man eine eigene Datenbank für das Testen erstellt.
        Um die Testdaten wieder zu löschen, ist es am besten die erstellten Collections der MongoDB zu löschen.

    \section{Troubleshooting}
        \subsection{Keine Verbindung zur Datenbank möglich}

%------------------------------------------------------------------------%
\end{document}